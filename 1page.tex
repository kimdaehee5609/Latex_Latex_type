%	-------------------------------------------------------------------------------
% 
%
%
%
%
%
%
%
%
%
%	-------------------------------------------------------------------------------

\documentclass[12pt,a4paper,oneside]{book}

		% --------------------------------- 페이지 스타일 지정
		\usepackage{geometry}
		\geometry{top 		=10em}
		\geometry{bottom		=10em}
		\geometry{left		=8em}
		\geometry{right		=8em}
		\geometry{headheight	=4em} % 머리말 설치 높이
		\geometry{headsep		=2em} % 머리말의 본문과의 띠우기 크기
		\geometry{footskip		=4em} % 꼬리말의 본문과의 띠우기 크기
% 		\geometry{showframe}
	
%		paperwidth 	= left + width + right (1)
%		paperheight 	= top + height + bottom (2)
%		width 		= textwidth (+ marginparsep + marginparwidth) (3)
%		height 		= textheight (+ headheight + headsep + footskip) (4)



		%	===================================================================
		%	package
		%	===================================================================
%			\usepackage[hangul]{kotex}				% 한글 사용
			\usepackage{kotex}						% 한글 사용
			\usepackage[unicode]{hyperref}			% 한굴 하이퍼링크 사용
			\usepackage{amssymb,amsfonts,amsmath}	% 수학 수식 사용

			\usepackage{scrextend}					% 
		
			\usepackage{enumerate}			%
			\usepackage{enumitem}			%
			\usepackage{tablists}			%	수학문제의 보기 등을 표현하는데 사용
										%	tabenum


		% ------------------------------ table 
			\usepackage{longtable}			%
			\usepackage{tabularx}			%

			\usepackage{setspace}			%
			\usepackage{booktabs}			% table
			\usepackage{color}				%
			\usepackage{multirow}			%
			\usepackage{boxedminipage}		% 미니 페이지
			\usepackage[pdftex]{graphicx}	% 그림 사용
			\usepackage[final]{pdfpages}	% pdf 사용
			\usepackage{framed}			% pdf 사용
			
			\usepackage{fix-cm}	
			\usepackage[english]{babel}

		% ------------------------------ TikZ picture
	
			\usepackage{tikz}				
			\usetikzlibrary{shapes,arrows,positioning}
			\usetikzlibrary{shadows}
			%\usetikzlibrary{positioning}

			\usepackage{tikzscale}
			\usepackage{adjustbox}

			

		% --------------------------------- page
			\usepackage{afterpage}		% 다음페이지가 나온면 어떻게 하라는 명령 정의 패키지


			\usepackage{blindtext}
	
		% --------------------------------- font 사용
			\usepackage{pifont}				%
			\usepackage{textcomp}
			\usepackage{gensymb}
			\usepackage{marvosym}






		% --------------------------------- 페이지 스타일 지정

		\usepackage[Bjornstrup]{fncychap}

		\usepackage{fancyhdr}
		\pagestyle{fancy}
		\fancyhead{} % clear all fields
		\fancyhead[LO]{\tiny \leftmark}
		\fancyhead[RE]{\tiny \leftmark}
		\fancyfoot{} % clear all fields
		\fancyfoot[LE,RO]{\large \thepage}
		%\fancyfoot[CO,CE]{\empty}
		\renewcommand{\headrulewidth}{1.0pt}
		\renewcommand{\footrulewidth}{0.4pt}
	
	
	
		% --------------------------------- 	section 스타일 지정
	
		\usepackage{titlesec}
		
		\titleformat*{\section}		{\large\bfseries}
		\titleformat*{\subsection}		{\normalsize\bfseries}
		\titleformat*{\subsubsection}	{\normalsize\bfseries}
		\titleformat*{\paragraph}		{\normalsize\bfseries}
		\titleformat*{\subparagraph}	{\normalsize\bfseries}
	
		\renewcommand{\thesection}		{\arabic{section}.}
		\renewcommand{\thesubsection}	{\thesection\arabic{subsection}.}
		\renewcommand{\thesubsubsection}{\thesubsection\arabic{subsubsection}}
		
		\titlespacing*{\section} 		{0pt}{1.0em}{1.0em}
		\titlespacing*{\subsection}	  	{0ex}{1.0em}{1.0em}
		\titlespacing*{\subsubsection}	{0ex}{1.0em}{1.0em}
		\titlespacing*{\paragraph}		{0ex}{1.0em}{1.0em}
		\titlespacing*{\subparagraph}	{0ex}{1.0em}{1.0em}
	
	%	\titlespacing*{\section} 		{0pt}{0.0\baselineskip}{0.0\baselineskip}
	%	\titlespacing*{\subsection}	  	{0ex}{0.0\baselineskip}{0.0\baselineskip}
	%	\titlespacing*{\subsubsection}	{6ex}{0.0\baselineskip}{0.0\baselineskip}
	%	\titlespacing*{\paragraph}		{6pt}{0.0\baselineskip}{0.0\baselineskip}
	

		% --------------------------------- recommend		섹션별 페이지 상단 여백
		\newcommand{\SectionMargin}{\newpage  \null \vskip 2cm}
		\newcommand{\SubSectionMargin}{\newpage  \null \vskip 2cm}
		\newcommand{\SubSubSectionMargin}{\newpage  \null \vskip 2cm}


	
		% ----------------------------- 장의 목차
		\usepackage{minitoc}
		\setcounter{minitocdepth}{1}    	% Show until subsubsections in minitoc
		\setlength{\mtcindent}{12pt} 		% default 24pt
	
	
		% --------------------------------- 	문서 기본 사항 설정
		\setcounter{secnumdepth}{3} 		% 문단 번호 깊이
		\setcounter{tocdepth}{3} 			% 문단 번호 깊이
		\setlength{\parindent}{0cm} 		% 문서 들여 쓰기를 하지 않는다.
		
		
		% --------------------------------- 	줄간격 설정
		\doublespace
%		\onehalfspace
%		\singlespace
		
		
% 	============================================================================== List global setting
%		\setlist{itemsep=1.0em}
	
% 	============================================================================== enumi setting

%		\renewcommand{\labelenumi}{\arabic{enumi}.} 
%		\renewcommand{\labelenumii}{\arabic{enumi}.\arabic{enumii}}
%		\renewcommand{\labelenumii}{(\arabic{enumii})}
%		\renewcommand{\labelenumiii}{\arabic{enumiii})}


	%	-------------------------------------------------------------------------------
	%		Vertical and Horizontal spacing
	%	-------------------------------------------------------------------------------
		\setlist[enumerate,1]	{ leftmargin=8.0em, rightmargin=0.0em, labelwidth=0.0em, labelsep=0.0em }
		\setlist[enumerate,2]	{ leftmargin=8.0em, rightmargin=0.0em, labelwidth=0.0em, labelsep=0.0em }
		\setlist[enumerate,3]	{ leftmargin=8.0em, rightmargin=0.0em, labelwidth=0.0em, labelsep=0.0em }
		\setlist[enumerate]	{ itemsep=1.0em, leftmargin=6.0ex, rightmargin=0.0em, labelwidth=0.0em, labelsep=4.0ex }


	%	-------------------------------------------------------------------------------
	%		Label
	%	-------------------------------------------------------------------------------
%		\setlist[enumerate,1]{ label=\arabic*., ref=\arabic* }
%		\setlist[enumerate,1]{ label=\emph{\arabic*.}, ref=\emph{\arabic*} }
%		\setlist[enumerate,1]{ label=\textbf{\arabic*.}, ref=\textbf{\arabic*} }   	% 1.
%		\setlist[enumerate,1]{ label=\textbf{\arabic*)}, ref=\textbf{\arabic*)} }		% 1)
		\setlist[enumerate,1]{ label=\textbf{(\arabic*)}, ref=\textbf{(\arabic*)} }	% (1)
		\setlist[enumerate,2]{ label=\textbf{\arabic*)}, ref=\textbf{\arabic*)} }		% 1)
		\setlist[enumerate,3]{ label=\textbf{\arabic*.}, ref=\textbf{\arabic*.} }		% 1.

%		\setlist[enumerate,2]{ label=\emph{\alph*}),ref=\theenumi.\emph{\alph*} }
%		\setlist[enumerate,3]{ label=\roman*), ref=\theenumii.\roman* }


% 	============================================================================== itemi setting


	%	-------------------------------------------------------------------------------
	%		Vertical and Horizontal spacing
	%	-------------------------------------------------------------------------------
		\setlist[itemize]{itemsep=0.0em}


	%	-------------------------------------------------------------------------------
	%		Label
	%	-------------------------------------------------------------------------------
		\renewcommand{\labelitemi}{$\bullet$}
		\renewcommand{\labelitemii}{$\cdot$}
		\renewcommand{\labelitemiii}{$\diamond$}
		\renewcommand{\labelitemiv}{$\ast$}		




		% --------------------------------- recommend  글자 색깔지정 명령
		\newcommand{\red}{\color{red}}			% 글자 색깔 지정
		\newcommand{\blue}{\color{blue}}		% 글자 색깔 지정
		\newcommand{\black}{\color{black}}		% 글자 색깔 지정
		\newcommand{\superscript}[1]{${}^{#1}$}

	
	
		% --------------------------------- 환경 정의 : 박스 치고 안의 글자 빨간색

			\newenvironment{BoxRedText}
			{ 	\setlength{\fboxsep}{12pt}
				\begin{boxedminipage}[c]{1.0\linewidth}
				\color{red}
			}
			{ 	\end{boxedminipage} 
				\color{black}
			}
			
%		\setmainhangulfont[BoldFont=HY견고딕]{한컴돋움}
%		\setsanshangulfont{HY견고딕}
%		\setmonohangulfont{한컴돋움}
			
			

% ------------------------------------------------------------------------------
% Begin document (Content goes below)
% ------------------------------------------------------------------------------
	\begin{document}
	
			\dominitoc
			
			\begin{titlepage}
			\centering
			\vspace*{2cm} 
			\rule {1.0\textwidth}{0.2mm}
			\huge  1 page style
			\\[-6mm]
			\rule {1.0\textwidth}{0.2mm}
			\vfill
			\Large 2015년 7월 13일
			\vfill
			\large 서영엔지니어링 김대희
			\vspace {2cm}
			\end{titlepage}



%			\tableofcontents
%			\listoffigures
%			\listoftables

			



% ================================================= chapter 	====================
	\newpage
	\chapter{ 1 page style의 정의}


		\newpage
		\begin{itemize}
		\item 타원의 선굵기 조정
		\end{itemize}

	% 	------------------------------------------------------------------- 1 PAGE
	%	itemize 사용
	% 	------------------------------------------------------------------- 

		\newpage
	
	
		\thispagestyle{empty}
		\begin{center}
		\Large
		\begin{tikzpicture}
		\node	[	shape=rounded rectangle, draw, 
					rounded rectangle arc length=90, 
					minimum width=120mm,
					minimum height=16mm,
					fill=lightgray!20] 
				{ \textbf{문 서 제 목} };
		\end{tikzpicture}
		\end{center}
	
		\begin{itemize}[	leftmargin=0.0em, 
						rightmargin=0.0em, 
						labelwidth=0.0em, 
						labelsep=2.0em,
						itemsep=0.0em ]
		\item[1.] \textbf{개요}\\
	밸브의 축(STEM)과 Disc가 수직으로 개폐되는 방식으로 개폐 시 축(STEM)은 수직운동을 하여 밸브 위쪽으로 Stroke 만큼 돌출된다. 
	주로 GATE, GLOBE 밸브에 적용된다.
	
		\item[2.] \textbf{문제점}\\
	밸브의 축(STEM)과 Disc가 수직으로 개폐되는 방식으로 개폐 시 축(STEM)은 수직운동을 하여 밸브 위쪽으로 Stroke 만큼 돌출된다. 
	주로 GATE, GLOBE 밸브에 적용된다.
	
		\item[3.] 대책	\\
	밸브의 축(STEM)과 Disc가 수직으로 개폐되는 방식으로 개폐 시 축(STEM)은 수직운동을 하여 밸브 위쪽으로 Stroke 만큼 돌출된다. 
	주로 GATE, GLOBE 밸브에 적용된다.
	
		\item[4.] 결론\\
	밸브의 축(STEM)과 Disc가 수직으로 개폐되는 방식으로 개폐 시 축(STEM)은 수직운동을 하여 밸브 위쪽으로 Stroke 만큼 돌출된다. 
	주로 GATE, GLOBE 밸브에 적용된다.
	
		\end{itemize}
	
	
	
	% 	------------------------------------------------------------------- 1 PAGE
	%	description style=nestline 사용
	% 	------------------------------------------------------------------- 

		\newpage
	
	
		\thispagestyle{empty}
		\begin{center}
		\Large
		\begin{tikzpicture}
		\node	[	shape=rounded rectangle, draw, 
					rounded rectangle arc length=90, 
					minimum width=120mm,
					minimum height=16mm,
					fill=lightgray!20] 
				{ \textbf{문 서 제 목} };
		\end{tikzpicture}
		\end{center}
	
		\begin{description}[	style=nextline,
							leftmargin=3ex, 
							rightmargin=0.0em, 
							itemsep=0.0em ]
		\item[1. 개요]
	밸브의 축(STEM)과 Disc가 수직으로 개폐되는 방식으로 개폐 시 축(STEM)은 수직운동을 하여 밸브 위쪽으로 Stroke 만큼 돌출된다. 
	주로 GATE, GLOBE 밸브에 적용된다.
	
		\item[2. 문제점 ]
	밸브의 축(STEM)과 Disc가 수직으로 개폐되는 방식으로 개폐 시 축(STEM)은 수직운동을 하여 밸브 위쪽으로 Stroke 만큼 돌출된다. 
	주로 GATE, GLOBE 밸브에 적용된다.
	
		\item[3. 대책]
	밸브의 축(STEM)과 Disc가 수직으로 개폐되는 방식으로 개폐 시 축(STEM)은 수직운동을 하여 밸브 위쪽으로 Stroke 만큼 돌출된다. 
	주로 GATE, GLOBE 밸브에 적용된다.
	
		\item[4. 결론]
	밸브의 축(STEM)과 Disc가 수직으로 개폐되는 방식으로 개폐 시 축(STEM)은 수직운동을 하여 밸브 위쪽으로 Stroke 만큼 돌출된다. 
	주로 GATE, GLOBE 밸브에 적용된다.
	
		\end{description}
	
	

	% 	------------------------------------------------------------------- 1 PAGE
	%	제목 바탕의 색깔
	% 	------------------------------------------------------------------- 

		\newpage
	
	
		\thispagestyle{empty}

		\begin{center}
		\Large
		\begin{tikzpicture}
		\node	[	shape=rounded rectangle, draw, 
					rounded rectangle arc length=90, 
					minimum width=120mm,
					minimum height=16mm,
					fill=blue!20] 
				{ \textbf{문 서 제 목} };
		\end{tikzpicture}
		\end{center}
	

	% 	-------------------------------------
		\begin{center}
		\Large
		\begin{tikzpicture}
		\node	[	shape=rounded rectangle, draw, 
					rounded rectangle arc length=90, 
					minimum width=120mm,
					minimum height=16mm,
					fill=green!20] 
				{ \textbf{문 서 제 목} };
		\end{tikzpicture}
		\end{center}

	% 	-------------------------------------
		\begin{center}
		\Large
		\begin{tikzpicture}
		\node	[	shape=rounded rectangle, draw, 
					rounded rectangle arc length=90, 
					minimum width=120mm,
					minimum height=16mm,
					fill=cyan!20] 
				{ \textbf{문 서 제 목 cyan} };
		\end{tikzpicture}
		\end{center}


	% 	-------------------------------------
		\begin{center}
		\Large
		\begin{tikzpicture}
		\node	[	shape=rounded rectangle, draw, 
					rounded rectangle arc length=90, 
					minimum width=120mm,
					minimum height=16mm,
					fill=magenta!20] 
				{ \textbf{문 서 제 목 magenta } };
		\end{tikzpicture}
		\end{center}

	% 	-------------------------------------
		\begin{center}
		\Large
		\begin{tikzpicture}
		\node	[	shape=rounded rectangle, draw, 
					rounded rectangle arc length=90, 
					minimum width=120mm,
					minimum height=16mm,
					fill=black!10] 
				{ \textbf{문 서 제 목 black} };
		\end{tikzpicture}
		\end{center}


	% 	-------------------------------------
		\begin{center}
		\Large
		\begin{tikzpicture}
		\node	[	shape=rounded rectangle, draw, 
					rounded rectangle arc length=90, 
					minimum width=120mm,
					minimum height=16mm,
					fill=lime!20] 
				{ \textbf{문 서 제 목 lime } };
		\end{tikzpicture}
		\end{center}


	% 	-------------------------------------
		\begin{center}
		\Large
		\begin{tikzpicture}
		\node	[	shape=rounded rectangle, draw, 
					rounded rectangle arc length=90, 
					minimum width=120mm,
					minimum height=16mm,
					fill=purple!20] 
				{ \textbf{문 서 제 목 purple} };
		\end{tikzpicture}
		\end{center}


	% 	-------------------------------------
		\begin{center}
		\Large
		\begin{tikzpicture}
		\node	[	shape=rounded rectangle, draw, 
					rounded rectangle arc length=90, 
					minimum width=120mm,
					minimum height=16mm,
					fill=teal!20] 
				{ \textbf{문 서 제 목 teal } };
		\end{tikzpicture}
		\end{center}

















% ------------------------------------------------------------------------------
% End document
% ------------------------------------------------------------------------------
\end{document}


